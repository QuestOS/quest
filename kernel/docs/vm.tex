\documentclass[twocolumn]{article}
\usepackage{url}

\newcommand\func[1]{{\path {#1}}}
\newcommand\type[1]{{\path {#1}}}
\newcommand\optn[1]{{\path {#1}}}
\newcommand\defn[1]{{\path {#1}}}
\newcommand\MSR[1]{{\path {#1}}}
\newcommand\bit[2]{{\tt {#1}.{#2}}}
\newcommand\inst[1]{{\tt {#1}}}
\newcommand\reg[1]{{\tt {\%{#1}}}}

\begin{document}

\section{Introduction}
The Quest kernel contains limited support for utilizing the primitives
of Intel's hardware virtualization technology (VT) scheme, called VMX.
The module is implemented in the \path{vm/vmx.c} file and is
accompanied by a virtual-8086 monitor in \path{vm/vm86.c} for the
emulation of real-mode code.  The specification for VMX is described
in Intel's System Programmer's Manual 3B.  The instructions associated
with VMX are described in Intel's Instruction Set Reference 2B.  Many
of the definitions and constants introduced in those manuals are
imported into Quest header files under the \path{include/vm/} folder.

\subsection{Configuration}
The VMX module may be enabled via the configuration option
\optn{USE_VMX} in the \path{config.mk} file.

\subsection{Design}

\section{Initialization}
\subsection{Enabling VMX}
When enabled, the initialization of the VMX module takes place during
normal module initialization.  Application Processors also perform
per-CPU initialization of virtual machine code after the boot process
has completed but prior to scheduler invocation.  The function
\func{vmx_global_init} is called once, and \func{vmx_processor_init}
is invoked once per CPU.  Each processor must perform hardware
initialization in order to use VMX. VMX has certain stringent
requirements on the state of the processor and is very strict about
checking.  These requirements are meant to be obtained by reading a
number of MSRs which describe the necessary and possible settings of
various control registers.  Most implementations require \bit{CR0}{NE}
be enabled, so that is done early on explicitly.  \bit{CR4}{VMX} is
toggled to enable the use of VMX instructions.  A page must be
designated and reserved as the \inst{vmxon} area.  After that basic
bit of initialization is complete, the module goes directly into the
initialization for the isolation mechanism that is being designed for
Quest.

\subsection{Creating a virtual machine}
\subsection{Memory}
The original VMX module created VMs that operated in real-mode at the
beginning.  However, for the isolation mechanism, there is no need to
ever enter real-mode.  In either case, a virtual machine control
structure (VMCS) must be allocated and reserved for use by the
processor.  A VMCS is a single page which is not intended to be
accessed directly by software, with one exception: the first 64-bits
must contain the zero-extended value of the MSR \MSR{IA32_VMX_BASIC}.
The physical address of the VMCS may be used with the \inst{vmclear}
and \inst{vmptrld} instructions to prepare and load a VMCS into the
processor.  Only a single VMCS may be loaded on a processor at a time.
In order to move a VMCS from one processor to another, it is necessary
to unload it and then load it onto the new processor.  Anytime a VMCS
is freshly loaded onto a processor, the \inst{vmlaunch} instruction
must be used in order to enter the VM.  The \inst{vmresume}
instruction can only be used to re-enter an already-launched VM.

In addition to the processor-specific information stored in the VMCS,
Quest has its own \type{virtual_machine} structure.  This stores the
physical address of the VMCS as well as some other useful state
information, including: whether the VM is in real-mode, whether the VM
has been ``launched,'' and the state of the general-purpose registers
in the VM.

\subsection{VMCS fields}
Once all memory has been allocated and the VM is loaded onto the
processor, it is possible to use the \inst{vmread} and \inst{vmwrite}
instructions to configure the various parameters of the virtual
machine.  There are an extensive number of these parameters, and a
full description can be found in the manual.  Roughly speaking, there
are parameters for: host state, guest state, event injection, and
VM-exit triggers.  All of the fields are given an index which can be
used with \inst{vmread} and \inst{vmwrite}.  In Quest these indices
are defined by constants prefixed by \defn{VMXENC_}.  While many of
the fields are 64-bit in size, we are only operating and emulating
32-bit mode, therefore we only deal with the least significant bits of
the fields.

\subsection{Guest state}
In Quest, we assume that the current state of the CPU is the starting
point for the virtual machine.  Therefore, the code proceeds to read
the various flag, control, and segment registers from the machine and
write those values into the guest fields.  VMX also requires that
certain ``hidden'' processor state be initialized in the VMCS guest
fields.  This includes the base, limit, and access rights for segment
selectors: things that are normally loaded from the GDT by the
processor, but must be manually loaded into fields here.

The general-purpose registers are not stored in the VMCS.  However,
the registers \reg{rip} (\reg{eip}) and \reg{rsp} (\reg{esp}) are
stored and loaded from the VMCS.  In addition, the
\inst{sysenter}-related MSRs are treated this way as well.  There is
also some additional processor-internal state which is exposed via the
\defn{ACTIVITY} and \defn{INTERRUPTIBILITY} fields.

\subsection{Exit reasons}
The VMCS can also be configured with flags that describe the
conditions under which the virtual machine exits and returns to
``root'' operation.  For example, there is a 32-bit \defn{EXCEPTION}
bitmap indexed by exception number where a 1-setting indicates that
the given exception should cause a VM exit.  The page fault exception
also has a few other fields for more specific behavior.

Similarly, there is a flag which controls whether external interrupts
cause VM exits, and one for the NMI.  These are part of the
``pin-based'' controls.  The ``processor-based'' controls are flags
for whether certain aspects of processor behavior should cause VM
exits.  For example, we disable VM exits for the access of \reg{cr3}
as well as instructions \inst{rdtsc} and \inst{rdpmc}.  VM exits for
MSR access can be controlled through the MSR-bitmap, which is also
enabled here.  Some of these controls are actually configured in
\func{vmx_start_vm}.

\section{Entering a virtual machine}

\subsection{The initial instruction pointer}
The function \func{vmx_enter_pmode_VM} saves a value of \reg{eip} with
a bit of inline assembly which \inst{call}s a label and pops the
stored value of \reg{eip} off the stack.  The program point of the
stored \reg{eip} corresponds to the point in the inline assembly after
the return of the \inst{call}.  Therefore, when the VM is launched,
this path of execution can be distinguished by clearing the register
storing the value of \reg{eip}.  In addition to saving \reg{eip}, it
is important to fork a stack at this point.  The VM will carry on
using the original stack, but the hypervisor must copy the stack and
have a separate one.  This ensures that there will be no interference
between the two.  Since the VM will begin operation inside this
function, the state of the stack up to this point is preserved; the
hypervisor's stack is put into effect immediately by carefully
assigning a new, functionally equivalent, value for \reg{esp} in the
new stack.  The VM is prepared with its initial values of \reg{eip}
and \reg{esp} before making the call to \func{vmx_start_VM}.

\subsection{Entry and preparations for eventual exit}
The function \func{vmx_start_VM} must prepare for VM entry by saving
the state of the host currently.  This is a similar process to the
preparation of guest state as described previously, except that there
are fewer fields to fill.  General-purpose registers are not managed
by VMX, therefore it is necessary to save host registers and restore
guest registers.  The host registers are stored on the host stack
using \inst{pusha}.  Again, we need to fork paths and save a value of
\reg{eip} to store in the VMCS field \defn{HOST_RIP}.  The guest
registers are loaded from memory by copying them onto the stack in
such a way that \inst{popa} can restore them all at once; just prior
to the usage of either \inst{vmlaunch} or \inst{vmresume}.

\subsection{Entry failure}
The specification defines a series of conditions that must be
fulfilled for successful VM entry.  There are a number of failure
modes depending on how far the process gets before aborting.  Late
failures can lead to a full VM exit.  Early failures are different,
the processor simply advances to the instruction following the
\inst{vmlaunch} or \inst{vmresume} and sets the zero and/or carry flag
according to various criteria.  However, the registers still contain
the values for the guest.  Therefore, it is necessary to carefully
restore the host registers -- especially \reg{esp} -- while also
checking the flag values.  The usual tactic of using \inst{pushf} does
not work: the value of \reg{esp} is available in the VMCS field
\defn{HOST_RSP} but to use \inst{vmread} would clobber the flags.
Therefore, the flags must be case-analyzed via branching, at which
point the value of \reg{esp} can be restored; followed by the
general-purpose registers which have been saved on the stack.  The
exact cause of the failure can be pin-pointed through certain VMCS
fields which contain error codes to be analyzed.

\section{Hypervisor}
\subsection{Handling VM exits}
In the case of partially or totally successful VM entry, the machine
enters a state that is designated as ``non-root operation.''  Upon VM
exit, the machine transfers control to the value stored in the VMCS
field \defn{HOST_RIP}, and switches to the stack value saved in
\defn{HOST_RSP}.  Because guest general-purpose registers are not
handled by VMX, the very first instruction we use is \inst{pusha} to
snapshot them on the host stack.  In order to continue operation and
save the guest registers into the \type{virtual_machine} structure, we
must restore the host registers.  We know where they are on the stack
-- adjacent to the snapshot of the guest registers.  Therefore, a
quick trick suffices: add 32 to \reg{esp}, \inst{popa} the host
registers, then load \reg{esp-64} into \reg{esi} and quickly copy the
stack values into the structure (the pointer has already been stored
into \reg{edi}).  The stack is then restored to the original point and
the host registers are once again popped, to restore any that were
clobbered by the operation thus far.

Once again, a number of VMCS fields have been prepared with codes that
explain the cause of the VM exit.  The hypervisor then case-analyzes
those codes in order to determine how to proceed.  For example,
\inst{cpuid} causes an unconditional VM exit.  Therefore, it is
necessary for the hypervisor to emulate the instruction.  It
manipulates the guest registers in the saved structure, and then
re-enters the VM at the instruction following \inst{cpuid}.
Similarly, when emulating real-mode, the use of virtual-8086 means
that general-protection faults are, in fact, requests for intervention
by the monitor.  Therefore, the function \func{vmx_vm86_handle_GPF} is
invoked to analyze and emulate any instruction that requires such
assistance.

\end{document}
